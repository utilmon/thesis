\chapter{Conclusion}

An electron source capable of sustaining high-intensity production of spin-polarized electrons is highly sought after for both current and future nuclear and high energy physics facilities, such as the Electron Ion Collider and the International Linear Collider \cite{wang2022high,wang2021_DesignHighCharge, ilc2019}.
Furthermore, electron microscopy techniques can utilize spin-polarized electrons to study magnetization in materials and nanostructures \cite{suzuki2010_RealTimeMagnetic,kuwahara2012_30kVSpinpolarizedTransmission,vollmer2003}.
A Negative Electron Affinity (NEA) activated GaAs in a high voltage electron gun is the best option to date to produce a highly spin-polarized electron beam at a high current.
However, extreme vaccum sensitivity of NEA layer is the main drawback that results in rapid Quantum Efficiency (QE) degradation over time; hence it has a short operational lifetime \cite{grames2011_ChargeFluenceLifetime, bae2018_RuggedSpinpolarizedElectron,bae2020_ImprovedLifetimeHigh,cultrera2020_LongLifetimePolarized}.
Traditional NEA activation layers such as Cs-O$_2$ and Cs-NF$_3$ form a monolayer that is chemically reactive and weakly bound to the surface \cite{kuriki2011_DarklifetimeDegradationGaAs,chanlek2014_DegradationQuantumEfficiency}.
Recent studies showed that robust solar-blind material Cs$_2$Te could also form NEA on GaAs due to a peculiar alignment of the electronic bands \cite{kuriki2015_GaAsPhotocathodeActivation,sugiyama2011_StudyElectronAffinity,uchida2014_STUDYROBUSTNESSNEAGAAS}.
%In this thesis, we activate GaAs with Cs$_2$Te and characterize various photocathode parameters, such as spectral response, lifetime, and spin polarization \cite{bae2018_RuggedSpinpolarizedElectron}. We extend this approach to optimize the growth recipe by using oxygen and Sb elements \cite{bae2020_ImprovedLifetimeHigh}. Lastly, we test Cs-Sb-O activated GaAs in a high voltage DC gun and operates for a high current extraction \cite{bae2022}.

In this thesis, we first activated GaAs with Cs$_2$Te to characterize various photocathode parameters. We obtained a similar spectral response to Ref.~\cite{kuriki2015_GaAsPhotocathodeActivation} and confirmed NEA activation on GaAs.
Auger electron spectroscopy was used to study the NEA surface elements.
QE at 532 nm was monitored while extracting current to characterize the lifetime. We achieved a factor of 5 improvements in the lifetime at 532 nm compared to the standard Cs-O activated GaAs.
A Mott polarimeter was used to characterize the degree of spin polarization. Despite a thicker NEA layer of Cs$_2$Te, the spin polarization was not affected.

Based on the heterojunction model between GaAs and Cs$_3$Sb, we were motivated to study activation of GaAs using Cs$_3$Sb. It was discovered that codepositing oxygen during the growth can increase the QE at 780 nm without a significant decrease in the photocathode lifetime. We obtained one order of magnitude improvement in a lifetime at 505 nm and a factor of 7 improvements in a lifetime at 780 nm. Similar to the previous study, spin polarization was minimally affected. Furthermore, a superlattice sample was activated with Cs-Sb-O and compared to the standard activated sample. 90\% spin polarization was preserved with a factor of 7 improvements in a lifetime at 780 nm.

Although lifetime improvements were demonstrated in the previous works, these photoemission measurements were done with a 10's of eV bias and 100's of nA beam current, which is significantly different from the harsh environment of an electron gun that operates on the scale of 100's of keV. Specifically, the amount of current extracted and the energy-dependent residual gas ionization cross-sections are orders of magnitude different \cite{grames2011_ChargeFluenceLifetime}. Furthermore, the primary QE degradation mechanism was chemical poisoning in our previous works as opposed to ion back bombardment \cite{cultrera2020_LongLifetimePolarized}.
We recommissioned a high voltage DC gun and operated Cs-Sb-O and Cs-O activated GaAs photocathodes in order to make a direct comparison of their performance at 200 keV beam energy at 1 mA average current --the ultimate test required to establish the photocathode performance under ``real life" conditions.
We observed spectral dependence on the lifetime improvement.
In particular, we saw a 45\% increase in the lifetime at 780 nm for Cs-Sb-O activated GaAs compared to Cs-O activated GaAs.

Chapter 5 presents photoemission simulations of Cu photocathode using a Boltzmann equation method that has capability to account nonequilibrium thermodynamics and multiphoton photoemission. Using a low absorbed fluence ($10^{-5} \textrm{ mJ/cm}^2$), the calculated photoemission parameters reproduced the experimentally measured QE and demonstrated good agreement of MTE with earlier, static equilibrium predictions due to the abundance of singly excited electrons. In contrast, for a high absorbed fluence ($1 \textrm{ mJ/cm}^2$), multiphoton absorbed electrons are no longer negligible especially near the threshold photon energy, thereby causing the growth of MTE. Since the minimum MTE is no longer the thermal energy and not achieved by the work function photon energy, a series of laser fluences were simulated to plot the maximum achievable brightness for a given number of extracted electrons in a 50 fs pulse. Our results illustrate the importance of multiphoton effects on beam brightness of photocathodes irradiated by high intensity laser.
%A dynamic photoemission calculation like the one presented here could be self-consistently coupled to a space charge dynamics tracking code for precision photoinjector modeling.
This work implies that such multiphoton effects are significant for other photocathodes as well for femtosecond applications, including GaAs-based photocathodes. 