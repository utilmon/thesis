\begin{abstract}

%An electron source capable of sustaining high-intensity production of spin-polarized electrons is highly desired for 
GaAs-based photocathodes are considered state-of-the-art for producing highly spin-polarized electron beams for accelerator and microscopy applications.
Negative Electron Affinity (NEA) activated surfaces are required to extract highly spin-polarized electron beams from GaAs-based photocathodes, but they suffer extreme sensitivity to poor vacuum conditions that result in rapid degradation of quantum efficiency (QE).
This thesis investigates unconventional activation recipes of GaAs using Te and Sb elements in addition to conventional Cs and O$_2$.
Photocathode parameters, such as spectral response, spin polarization, and lifetime, are characterized for various growth recipes. To conclude the thesis, GaAs photocathodes with an optimal activation recipe (Cs-Sb-O) were characterized in a high voltage environment for a high current extraction.

%In this thesis, we activate GaAs with Cs$_2$Te and characterize various photocathode parameters, such as spectral response, lifetime, and spin polarization \cite{bae2018_RuggedSpinpolarizedElectron}. We extend this approach to optimize the growth recipe by using oxygen and Sb elements \cite{bae2020_ImprovedLifetimeHigh}. Lastly, we test Cs-Sb-O activated GaAs in a high voltage DC gun and operates for a high current extraction \cite{bae2022}.

\end{abstract}
